\chapter {Cap\'itulo Introductorio}\label{Introduccion}

\paragraph{En este cap\'itulo van:}

\paragraph{Introducci\'on}
\paragraph{Problema, motivaci\'on}
\paragraph{Objetivos generales y espec\'ificos}

\paragraph{Escritos que podr\'ian ser reutilizados:}

\paragraph{\textit{Las herramientas mediadoras de redes sociales permiten a sus usuarios compartir publicaciones entre s\'i. Los autores de dichos mensajes escriben sobre sus vidas, comparten opiniones sobre diferentes t\'opicos y discuten problemas actuales. A medida que m\'as y m\'as usuarios publican mensajes acerca de los productos y servicios que utilizan, estos sitios de mediaci\'on se convierten en una fuente valiosa de descripci\'on de las opiniones y sentimientos de las personas. Tales datos pueden ser eficientemente utilizados para estudios sociales (Pak y Paroubek, 2010). Estos datos pueden ser convertidos en informaci\'on que permita estimar la sensaci\'on general acerca de un determinado t\'opico o entidad.}}

\paragraph{\textit{El an\'alisis de sentimientos fue objeto de estudio a diferentes niveles de granularidad. Inicialmente se ocuparon del an\'alisis de cr\'iticas acerca de productos espec\'ificos, con el objetivo de clasificarlos como ``recomendables'' o ``no recomendables'', a partir de asociaciones sem\'anticas de las oraciones (Turney, 2002; Pang y Lee, 2004). Otros estudios clasificaron los sentimientos al nivel de oraciones, defini\'endolas como ``positivas'' o ``negativas'' (Hu y Liu, 2004; Kim y Hovy, 2004). M\'as recientemente, atra\'idos por la variedad y diversidad de los datos generados, fueron objeto de an\'alisis los mensajes compartidos en mediadores de redes sociales, los cuales tienen una longitud limitada y son generados en tiempo real. Algunos trabajos que abordaron estas fuentes son los de Pak y Paroubek (2010), que clasificaron los mensajes de Twitter como ``positivos'', ``negativos'' o ``neutrales''; Go y otros (2009), que utilizaron los tuits con emoticones para clasificarlos como ``positivos'' o ``negativos''; y Saralegi Urizar y San Vicente Roncal (2012), que clasificaron las publicaciones de Twitter en espa\~nol como ``muy positivas'', ``positivas'', ``neutrales'', ``negativas'', ``muy negativas'' o ``ninguna de las anteriores''.}}

\paragraph{Para clasificar como positivos o negativos los sentimientos expresados en un documento o mensaje, existen dos principales enfoques com\'unmente utilizados: la estrategia basada en l\'exicos, y la estrategia de aprendizaje de m\'aquina (Zhang y otros, 2011), los cuales son abordados en las siguientes secciones. }