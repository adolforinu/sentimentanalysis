\chapter{Conclusiones y trabajos futuros}\label{conclusiones}

En este trabajo presentamos la aplicaci\'on de t\'ecnicas conocidas de clasificaci\'on de textos, siguiendo estrategias basadas en l\'exico as\'i como de aprendizaje de m\'aquina, aplicadas al lenguaje coloquial distintivo de una poblaci\'on biling\"ue como la del Paraguay.
\newline

Abordamos el estudio del an\'alisis de sentimientos, enfocando concretamente el esfuerzo en la detecci\'on de polaridad de una entidad de texto. El escenario presentado consiste en la interacci\'on de los usuarios de un servicio de telefon\'ia m\'ovil a trav\'es del sitio de la red social Twitter.
\newline

Este an\'alisis implica el seguimiento de las etapas de un proceso de miner\'ia de datos, resultando principalmente fundamentales el preprocesamiento de datos, el modelado y la evaluaci\'on de resultados. Finalmente fue establecida una comparaci\'on entre los diferentes m\'etodos utilizados. Presentamos luego, en la siguiente secci\'on, las conclusiones de este trabajo en funci\'on a los objetivos inicialmente trazados.

\section{Evaluaci\'on de objetivos propuestos}

Dados los objetivos espec\'ificos trazados inicialmente en el cap\'itulo introductorio de este trabajo, comentamos brevemente cada uno de ellos a manera de evaluaci\'on general y de conclusi\'on del mismo.

\begin{itemize}
\item \textbf{Obtener los datos necesarios para iniciar el an\'alisis del escenario presentado en esta propuesta.}
\newline

La extracci\'on de datos es un proceso que puede ser automatizado mediante el uso de la herramienta disponible para el efecto, relacionada al propio sitio que representa la fuente de datos. La limitaci\'on relativa a la cantidad de datos recuperables por cada consulta, es resuelta mediante la generaci\'on de cada una de ellas entre intervalos de tiempo. Esto permite que la cantidad de datos recolectados pueda ser definida sin mayores limitaciones o inclusive la realizaci\'on de un an\'alisis continuo sobre un flujo de datos que sea actualizado constantemente.
\newline

\item \textbf{Analizar las tareas y los algoritmos de miner\'ia existentes; modelarlos de manera que sean aplicables al problema presentado.}
\newline

Abordamos las dos estrategias existentes para el tratamiento del problema de clasificaci\'on de textos: la basada en l\'exico y la basada en aprendizaje de m\'aquina. Para ambas estrategias, el modelado determinado consiste en tratar a cada tuit como una instancia de evaluaci\'on, y tambi\'en de entrenamiento en el caso de los clasificadores de m\'aquina. Luego, el enfoque consiste en tratar las palabras como $n$-gramas, es decir una secuencia de $n$ palabras. La forma m\'as utilizada es la de unigramas, o bien palabra por palabra; sin embargo, fue demostrado tambi\'en en el preprocesamiento que algunos bigramas son m\'as eficientemente utilizados de forma unificada como atributo. Este modelo es aplicable a ambos tipos de estrategia, permitiendo centrar los esfuerzos posteriores en la asigncaci\'on de los par\'ametros relativos a cada algoritmo en particular.
\newline

\item \textbf{Conceptualizar un diccionario de palabras del lenguaje hablado en Paraguay, clasificadas por su orientaci\'on de polaridad como positivas o negativas, que pueda ser utilizado como referencia para futuros trabajos de an\'alisis de sentimientos.}
\newline

Evaluaci\'on del objetivo.
\newline

\item \textbf{Discutir las ventajas y desventajas de la utilizaci\'on del diccionario, adem\'as de las consideraciones pertinentes para su construcci\'on.}
\newline

Evaluaci\'on del objetivo.
\newline

\item \textbf{Establecer una serie de reglas de preprocesamiento de datos con criterio unificado, que sea aplicable de manera gen\'erica a todas las t\'ecnicas de clasificaci\'on presentadas.} 
\newline

Evaluaci\'on del objetivo.
\newline

\item \textbf{Desarrollar una t\'ecnica que sea implementada a trav\'es de una herramienta que analice autom\'aticamente los posibles sentimientos de un usuario, expresados en el lenguaje h\'ibrido paraguayo que conjunta t\'erminos de los idiomas espa\~nol y guaran\'i.}
\newline

Evaluaci\'on del objetivo.
\newline

\item \textbf{Proponer una estructura de evaluaci\'on del enfoque utilizado en este trabajo, en funci\'on a los diferentes tipos de entradas que sean requeridas por cada algoritmo.}
\newline

Evaluaci\'on del objetivo.
\newline
\end{itemize}

\section{Trabajos futuros}

Aqu\'i creamos una lista de trabajos futuros.
\newline

Alg\'un p\'arrafo de cierre.