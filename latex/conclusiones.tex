\chapter{Conclusiones y trabajos futuros}\label{conclusiones}

En este trabajo presentamos la aplicaci\'on de t\'ecnicas conocidas de clasificaci\'on de textos, siguiendo estrategias basadas en l\'exico as\'i como de aprendizaje de m\'aquina, aplicadas al lenguaje coloquial distintivo de una poblaci\'on biling\"ue como la del Paraguay.
\newline

Abordamos el estudio del an\'alisis de sentimientos, enfocando concretamente el esfuerzo en la detecci\'on de polaridad de una entidad de texto. El escenario presentado consiste en la interacci\'on de los usuarios de un servicio de telefon\'ia m\'ovil a trav\'es del sitio de la red social Twitter.
\newline

Este an\'alisis implica el seguimiento de las etapas de un proceso de miner\'ia de datos, resultando principalmente fundamentales el preprocesamiento de datos, el modelado y la evaluaci\'on de resultados. Finalmente fue establecida una comparaci\'on entre los diferentes m\'etodos utilizados. Presentamos luego, en la siguiente secci\'on, las conclusiones de este trabajo en funci\'on a los objetivos inicialmente trazados.

\section{Evaluaci\'on de objetivos propuestos}

Dados los objetivos espec\'ificos trazados inicialmente en el cap\'itulo introductorio de este trabajo, comentamos brevemente cada uno de ellos a manera de evaluaci\'on general y de conclusi\'on del mismo.

\begin{itemize}
\item \textbf{Obtener los datos necesarios para iniciar el an\'alisis del escenario presentado en esta propuesta.}
\newline

La extracci\'on de datos es un proceso que puede ser automatizado mediante el uso de la herramienta disponible para el efecto, relacionada al propio sitio que representa la fuente de datos. La limitaci\'on relativa a la cantidad de datos recuperables por cada consulta, es resuelta mediante la generaci\'on de cada una de ellas entre intervalos de tiempo. Esto permite que la cantidad de datos recolectados pueda ser definida sin mayores limitaciones o inclusive la realizaci\'on de un an\'alisis continuo sobre un flujo de datos que sea actualizado constantemente.
\newline

\item \textbf{Analizar las tareas y los algoritmos de miner\'ia existentes; modelarlos de manera que sean aplicables al problema presentado.}
\newline

Abordamos las dos estrategias existentes para el tratamiento del problema de clasificaci\'on de textos: la basada en l\'exico y la basada en aprendizaje de m\'aquina. Para ambas estrategias, el modelado determinado consiste en tratar a cada tuit como una instancia de evaluaci\'on, y tambi\'en de entrenamiento en el caso de los clasificadores de m\'aquina. Luego, el enfoque consiste en tratar las palabras como $n$-gramas, es decir una secuencia de $n$ palabras. La forma m\'as utilizada es la de unigramas, o bien palabra por palabra; sin embargo, fue demostrado tambi\'en en el preprocesamiento que algunos bigramas son m\'as eficientemente utilizados de forma unificada como atributo. Este modelo es aplicable a ambos tipos de estrategia, permitiendo centrar los esfuerzos posteriores en la asigncaci\'on de los par\'ametros relativos a cada algoritmo en particular.
\newline

\item \textbf{Conceptualizar un diccionario de palabras del lenguaje hablado en Paraguay, clasificadas por su orientaci\'on de polaridad como positivas o negativas, que pueda ser utilizado como referencia para futuros trabajos de an\'alisis de sentimientos.}
\newline

Confeccionamos el diccionario mencionado a partir del conjunto de muestra, que luego fue utilizado en un enfoque simple de clasificaci\'on basado en l\'exico con resultados aceptables. Las listas finales de palabras est\'an constituidas por 180 entradas positivas y por 634 entradas negativas, las cuales pueden verse listadas en detalle en el Ap\'endice \ref{Anexos}. Durante el proceso de confecci\'on pudimos observar adem\'as, que no solamente las palabras en espa\~nol y el guaran\'i componen el corpus del lenguaje abordado, sino que adem\'as existe una cantidad de palabras diferentes en ingl\'es mayor a la cantidad de palabras en guaran\'i y una peque\~na cantidad de palabras en portugu\'es (Secci\'on \ref{sec:categorizacion}). Esto implica que el an\'alisis de estrategias basadas en l\'exico no puede limitarse al espa\~nol y al guaran\'i, y que dicha consideraci\'on puede ser \'util adem\'as para paliar las dificultades que representan las ocurrencias \'unicas de las palabras en lenguajes diferentes, es decir, que no se repiten varias veces para los clasificadores de m\'aquina.
\newline

\item \textbf{Discutir las ventajas y desventajas de la utilizaci\'on del diccionario, adem\'as de las consideraciones pertinentes para su construcci\'on.}
\newline

La ventaja del diccionario consiste en que se puede contar con una lista de palabras y formas b\'asicas, que sean independientes a un conjunto determinado de entrenamiento y por ende, que permitan identificar a cada uno de los atributos de una instancia de evaluaci\'on como conocidos. Por otro lado, la desventaja pasa principalmente por la dependencia de contexto de la mayor\'ia de las palabras, como para clasificarlas por polaridad sin conocimiento previo del contenido a ser evaluado.
\newline

\item \textbf{Establecer una serie de reglas de preprocesamiento de datos con criterio unificado, que sean aplicables de manera gen\'erica a todas las t\'ecnicas de clasificaci\'on presentadas.} 

Ocho reglas de preprocesamiento fueron establecidas como tratamiento previo a los textos para los procedimientos de clasificaci\'on. Se tratan los v\'inculos de Internet, las palabras extendidas, la codificaci\'on de los tuits, los emoticones, las convenciones de lenguaje de la fuente de datos, la acentuaci\'on de palabras, los s\'imbolos y la inclusi\'on de bigramas. Los detalles de dichas reglas se encuentra en la Secci\'on \ref{procedimiento}, y ellas pueden ser utilizadas independientemente al lenguaje al que pertenezcan las palabras tratadas y al algoritmo de clasificaci\'on que sea utilizado posteriormente. 
\newline

\item \textbf{Proponer una estructura de evaluaci\'on del enfoque utilizado en este trabajo, en funci\'on a los diferentes tipos de entradas que sean requeridas por cada algoritmo.}
\newline

Un conjunto de tres m\'etricas distintas de evaluaci\'on fueron utilizadas, de las cuales dos de ellas (precisi\'on y exhaustividad) ofrecen descripciones diferentes de la utilidad de los resultados, mientras que la tercera consiste en una ponderaci\'on de las anteriores (medida-F). Adem\'as, el proceso de evaluaci\'on fue realizado sistem\'aticamente sobre conjuntos separados de entrenamiento y de evaluaci\'on, as\'i como el proceso de entrenamiento fue llevado a cabo sobre cargas con igual distribuci\'on de clases (balanceadas) y con distribuciones desiguales (desbalanceadas).
\newline

\item \textbf{Valorar la validez de la propuesta presentada a trav\'es de una comparaci\'on de resultados con trabajos y enfoques similares.}
\newline

Este comentario ser\'a agregado luego del desarrollo de la secci\'on correspondiente en el cap\'itulo de Resultados.
\newline
\end{itemize}

\section{Trabajos futuros}

En funci\'on a las conclusiones obtenidas y con el objetivos de futuras mejores, presentamos a continuaci\'on una serie de propuestas que podr\'ian dar continuidad al trabajo.

\begin{itemize}
\item Efectuar un an\'alisis que trate grados de polaridad (escalas m\'as extendidas de positivismo y negativismo) y objetos de referencia de los sentimientos.
\item Confeccionar un diccionario de entrada m\'as extendido a partir de corpus de texto reales que consideren todos los lenguajes identificados.
\item Evaluar una estructura multicapas conjuntando los diferentes algoritmos utilizados con el objetivo de elevar el rendimiento.
\item Incluir funciones de autoentrenamiento a los clasificadores.
\item Ampliar la aplicaci\'on de aspectos de NLP al problema tratado.
\end{itemize}

Finalmente, tras la bibliograf\'ia, presentamos en los ap\'endices algunos detalles que ofrecen mayor profundizaci\'on de aspectos espec\'ificos del trabajo.