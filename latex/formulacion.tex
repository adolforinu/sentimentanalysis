\chapter{Formulaci\'on del problema propuesto}\label{formulacion}

\paragraph{En este cap\'itulo van:}
\paragraph{Introducci\'on (Presentaci\'on del caso de an\'alisis)}
\paragraph{Modelo conceptual (selecci\'on de atributos, etc)}
\paragraph{Objetivos y criterios de \'exito}
\paragraph{Caracterizaci\'on de Twitter}
\paragraph{Extracci\'on de datos, dise\~no de pruebas (Implementaciones, balance de carga, etc)}

\section{Caracterizaci\'on de mensajes en Twitter}
\paragraph{Los mensajes compartidos en Twitter poseen varias caracter\'isticas \'unicas, que los diferencian de los dem\'as corpus de datos citados previamente. (Go y otros, 2009) identifican cuatro de ellas:}

\begin{itemize}
\item Longitud m\'axima: Cada publicaci\'on puede estar compuesta por una cantidad m\'axima de 140 caracteres, incluyendo signos de puntuaci\'on y s\'imbolos.
\item Alta disponibilidad: La API de Twitter permite obtener tuits libremente, a trav\'es de su funci\'on de b\'usqueda. Mediante esta funci\'on es posible recolectar una cantidad de tuits considerable para conformar los conjuntos de entrenamiento y de evaluaci\'on.
\item Modelo del lenguaje: Los mensajes de Twitter pueden generados desde diversos dispositivos, incluyendo los tel\'efonos celulares. Esto conlleva a que los errores ortogr\'aficos y expresiones informales se hagan mucho m\'as frecuentes en comparaci\'on a otros dominios.
\item Dominio: Los tuits se refieren a diversos temas, lo cual es relevante puesto que un gran porcentaje de estudios pasados fueron centrados en un dominio espec\'ifico.
\end{itemize}

\paragraph{Existen adem\'as ciertas convenciones de lenguaje de los tuits:}
\begin{itemize}
\item ``RT'' es un acr\'onimo de retuit, que se coloca delante de un mensaje para indicar que el mismo est\'a siendo repetido o compartido.
\item El caracter ``\#'' es utilizado para marcar, organizar y clasificar tuits por temas o categor\'ias.
\item El caracter ``@'' es empleado para referirse a una cuenta por su nombre de usuario.
\end{itemize}
\paragraph{Es frecuente adem\'as que los tuits vayan acompa\~nados de emoticones y v\'inculos de Web.}
